\documentclass[10pt]{beamer}
%\usetheme{Amsterdam}
\usetheme{Darmstadt}
%\usecolortheme{beaver}

%Include all packages
%\ProvidesPackage{./tex/mySlideShowStyle}
%Packages pour les langues
\usepackage[utf8]{inputenc}
\usepackage[frenchb]{babel}
\usepackage[T1]{fontenc}
\usepackage{float}
\usepackage{lmodern}
\usepackage{multimedia}

%Package pour la mise en forme
\usepackage{parallel}
\usepackage{setspace}
\usepackage{wrapfig}

%Packages pour les liens dynamiques
\usepackage{hyperref}
\usepackage{multicol}
%\usepackage{underscore}

%Packages pour les formules de Maths
\usepackage{amsmath}
\usepackage{amsfonts}
\usepackage{amssymb}
\usepackage{mathrsfs}
\usepackage{mathtools}
\usepackage{mathabx}
\usepackage{pseudocode}
\usepackage{fancybox}

%Packages des images
\usepackage{graphicx}
%\usepackage{subfigure}
\usepackage{subfig}
\usepackage{wrapfig}
\usepackage{caption}
%\usepackage{subcaption}

%Packages pour les string
\usepackage{xstring}

%Packages pour les listes
%\usepackage{enumitem}
\usepackage{moreverb}



%%%%%%%%%%%%%%%%%
\author{Nuno Guedelha}
\title{Implémentation et application d'un algorithme de Dynamique Hybride}
%\setbeamercovered{transparent} 
%\setbeamertemplate{navigation symbols}{} 
%\logo{} 
\institute{LAAS-CNRS}
\date{} 
%\subject{} 

\setbeamertemplate{footline}{
\leavevmode%
\hbox{\hspace*{-0.06cm}
\begin{beamercolorbox}[wd=.2\paperwidth,ht=2.25ex,dp=1ex,center]{author in head/foot}%
	\usebeamerfont{author in head/foot}\insertshortauthor%~~(\insertshortinstitute)
\end{beamercolorbox}%
\begin{beamercolorbox}[wd=.55\paperwidth,ht=2.25ex,dp=1ex,center]{section in head/foot}%
	\usebeamerfont{section in head/foot}\insertshorttitle
\end{beamercolorbox}%
\begin{beamercolorbox}[wd=.25\paperwidth,ht=2.25ex,dp=1ex,right]{section in head/foot}%
	\usebeamerfont{section in head/foot}\insertshortdate{}\hspace*{2em}
	\insertframenumber{} / \inserttotalframenumber\hspace*{2ex}
\end{beamercolorbox}}%
\vskip0pt%
}

\begin{document}


%==========================================================
%=============== variable fichier de figures ==============================

\newcommand{\myFiguresFile}{}
\newcommand{\setmyFiguresFile}[1]{\renewcommand{\myFiguresFile}{#1}}

%=============== include 1 figure =========================================

\ifx \incFig \undefined
\def \incFig [#1]#2{\includegraphics[width=#2, page=#1]{figs/\myFiguresFile}}
\fi

%=============== Display 1 figure =========================================

\ifx \dispFig \undefined
\def \dispFig [#1]#2#3#4#5%
{
\begin{figure}[#1]
  \begin{center}
  \includegraphics[width=#3, page=#2]{figs/\myFiguresFile}
  \IfStrEq{#4}{}{}{%
    \caption{#4}  % legende
    \label{#5}    % pour citer le numéro de figure
  }
  \end{center}
\end{figure}
}
\fi

%=============== 2 sous-figures alignées horizontalement =================

\ifx \dispTwoFig \undefined
\def \dispTwoFig [#1]#2#3#4#5#6#7%
{
\begin{figure}[#1]
\begin{center}
  \subfloat[#3 \label{#7.a}]{\includegraphics[width=7cm, page=#2]{figs/\myFiguresFile}}\hspace{1cm}
  \subfloat[#5 \label{#7.b}]{\includegraphics[width=7cm, page=#4]{figs/\myFiguresFile}}\hspace{1cm}
  \caption{#6}  % legende \\
  \label{#7} % pour citer le numéro de figure
\end{center}
\end{figure}
}
\fi

%=============== 3 sous-figures alignées horizontalement =================

\ifx \dispThreeFig \undefined
\def \dispThreeFig [#1]#2#3#4#5#6#7#8#9%
{
\begin{figure}[#1]
\begin{center}
  \subfloat[#3 \label{#9.a}]{\includegraphics[width=5cm, page=#2]{figs/\myFiguresFile}}\hspace{1cm}
  \subfloat[#5 \label{#9.b}]{\includegraphics[width=5cm, page=#4]{figs/\myFiguresFile}}\hspace{1cm}
  \subfloat[#7 \label{#9.c}]{\includegraphics[width=5cm, page=#6]{figs/\myFiguresFile}}\\
  \caption{#8}  % legende
  \label{#9}    % pour citer le numéro de figure
\end{center}
\end{figure}
}
\fi

%=============== 2 ou 3 colonnes alignée horizontalement ==================

\ifx \minipages \undefined
\def \minipages [#1]#2#3#4#5#6#7#8%
{
\begin{minipage}[#2]{#3\textwidth}
  #6
\end{minipage}
\begin{minipage}[#2]{#4\textwidth} \hfill
  #7
\end{minipage}
\IfStrEq{#1}{3}%
{
\begin{minipage}[#2]{#5\textwidth} \hfill
  #8
\end{minipage}
}
}
\fi

%=============== exemples ==================================================

%\begin{minipage}{.3\textwidth} \hfill
%  \begin{align*}
%  D_{O} = \lbrace &\textbf{d}_{Ox}, \textbf{d}_{Oy}, \textbf{d}_{Oy}, \\
%  &\textbf{d}_{x}, \textbf{d}_{y}, \textbf{d}_{z} \rbrace \subset M^{6}
%  \end{align*}
%\end{minipage}
%\begin{minipage}{.4\textwidth} \hfill
%  \begin{tabbing}
%  \= $\textbf{d}_{Ox}$ \= vecteur unitaire de rotation autour de $O_{x}$\\
%  \> $\textbf{d}_{Oy}$ \> vecteur unitaire de rotation autour de $O_{y}$\\
%  \> $\textbf{d}_{Oz}$ \> vecteur unitaire de rotation autour de $O_{z}$\\
%  \> $\textbf{d}_{x}$  \> vecteur unitaire de translation le long de $O_{x}$\\
%  \> $\textbf{d}_{y}$  \> vecteur unitaire de translation le long de $O_{y}$\\
%  \> $\textbf{d}_{z}$  \> vecteur unitaire de translation le long de $O_{z}$\\
%  \end{tabbing}
%\end{minipage}

%=============== autres macro textuelles ===================================

\newcommand{\cad}[0]{\textnormal{c'est à dire }}

\newcommand{\fd}[0]{\emph{fd}}

\newcommand{\mfd}[0]{\mathit{fd}}

\newcommand{\valTextwidth}[0]{\thetextwidth}

\newcommand{\valTextwidthUnit}[1]{\printinunitsof{#1}\prntlen{\textwidth}}

\newcommand{\valInUnit}[2]{\printinunitsof{#1}\prntlen{#2}}

%affichage des largeurs de zone texte
%\usepackage{layouts}
%\printinunitsof{cm}\prntlen{\textwidth}

\newcommand{\newglossdef}[3]
{\newglossaryentry{#1}%
{%
  name={#2},%
  description={#3}%
}}

%	\begin{frame}[allowframebreaks]{Title}
%	...
%	\framebreak
%	...
%	\end{frame}


%==========================================================

\begin{frame}
\titlepage
\end{frame}
%==========================================================

\begin{frame}{Plan}
\tableofcontents[part=01]
\tableofcontents[part=02]
\end{frame}
%==========================================================

\part{}

\section{Introduction}

\begin{frame}
  \frametitle{Contexte du stage}
  \framesubtitle{}
  \begin{block}{Projet professionnel}
  \begin{itemize}
    \item première expérience dans l'embarqué et le temps réel
    \item nouveau cap: la robotique mobile et le Master 2 IRR
    \item Objectif: chercheur en Robotique (entreprise ou laboratoire)
  \end{itemize}
  \end{block}
  \begin{block}{Le LAAS et Gepetto}
  \begin{itemize}
    \item Gepetto: expertise en analyse, génération de mouvements  
    \item Approche fondamentale, modélisation dynamique, contrôle, planification de mouvements
    \item {Intégration dans des packages open source \\
          $\hookrightarrow$ metapod: librairie de modélisation dynamique}
    \note{}
  \end{itemize}
  \end{block}
\end{frame}

\begin{frame}
  \frametitle{Objectifs}
  
	Implémenter un algorithme de dynamique hybride dans la librairie metapod:
	\begin{itemize}
		\item algorithme optimisé décrit par Roy Featherstone (“Rigid Body Dynamics Algorithm",2008)
\note{Roy Featherstone: Chercheur spécialisé en modélisation dynamique de robots. Ph.D. à Edinburgh -> univ Australian National -> Professeur invité à l'IIT}
		\item Formalisme de Newton-Euler, Algèbre Spatiale (mécanique des torseurs),
\note{vs formalisme Lagrange: effort généralisé sur chaque articulation en termes d'NRJ}
		\item metapod, l'algèbre spatiale, C++ template méta-programmation
  \end{itemize}
  	Analyser et appliquer les optimisations de Featherstone, combiner à Eigen:
  	\begin{itemize}
		\item Eigen, librairie templatée
		\item optimisations spécifiques à algo dyn hybride (réduire dimension du système à résoudre)
		\item optimisation branch sparsity (structure creuse de la matrice d'inertie H)
  \end{itemize}
		
  Réflexion sur l'application de la Dynamique Hybride
  	\begin{itemize}
		\item modélisation d'articulation flexibles ("compliant")
		\item commande de robot ayant une base flottante
  \end{itemize}
	
\end{frame}
%==========================================================

\section{Modélisation Dynamique}

\begin{frame}[allowframebreaks]
  \frametitle{}
  
  Indispensable dans certains cas:
  	\begin{itemize}
  	\item robot articulé, membres à l'inertie non négligeable
  	\item robot sans base fixe ou roulante (garantir l'appui au sol et l'équilibre)
  	\end{itemize}
  	
  On traite dans cet exposé: arbres cinématiques, pas les boucles
  
  Comment poser le problème dynamique. Equation de mouvement (notations):
  	\begin{description}
	  \item[$model$ :] le modèle dynamique du système à multi-corps rigides
	  \item[$q, \dot{q}, \ddot{q}$ :] vecteurs de position, vitesse, accélération des articulations du système
	  \item[$\tau$ :] forces/couples moteurs (internes) appliqués aux articulations
	  \item[$f^{ext}$ :] forces de contrainte extérieures (force de gravitation, coriolis, forces de contact, ...)
	\end{description}
	
	Les 3 types de problème dynamique:
  \begin{itemize}
  	\item dynamique directe $\longrightarrow$ trouver les accélérations induites par les forces appliquées aux articulations \\
	      Application: simulation \\
				\begin{equation}
				\ddot{q} = \mathrm{FD}(model,\bsy{q,\dot{q},\tau})
				\end{equation}
  \item dynamique inverse $\longrightarrow$ trouver les forces à appliquer aux accélérations pour produire les articulations voulues \\
	      Application:  contrôle, composante de dynamique directe ou inverse. \\
	      Utilité du calcul des couples actionneurs robot humanoïde: \\
        \begin{itemize}
        \item commande en couple ou
        \item estimation du centre de pression $\longrightarrow$ injecté dans le PC (preview control) $\longrightarrow$ générer trajectoire CoM
        \end{itemize}
				\begin{equation}
				\mathbf{\tau} = \mathrm{ID}(model,\bsy{q,\dot{q},\ddot{q}})
				\end{equation}
  \item dynamique hybride $\longrightarrow$ généralisation de de la dynamique directe et inverse \\
\note{pour chaque articulation, on connaît soit le couple (art. $fd$) soit l'accélération (art. $id$). L'algo calcule toutes les inconnues ($fd$ $\longrightarrow$ FD, $id$ $\longrightarrow$ ID)}
        Applications: simulation avec contraintes suppl sur une art. (excitation) \\
                    simulation simplifiée (certaines accélérations connues) \\
                    modéliser système à articulations flexibles \\
                    robot sans base fixe (la base flottante est l'articulation $fd$, couple appliqué est connu et nul, l'algo donne son accélération
%\note{Pourrait améliorer l'estimation de la position du centre de pression (ZMP) => utilisé par le filtre dynamique pour corriger la trajectoire du CoM du robot humanoïde
%                    $q,\dot{q},\ddot{q} = \mathrm{IK}(modele_ponctuel, c, \dot{c}, \ddot{c}, X^f)$
%                    $f,\tau = \mathrm{ID}(modele_complet,q,\dot{q},\ddot{q})$ $\longrightarrow CoP^MB$
%                    $\delta CoM = PreviewControl(\delta CoP) de Kajita$}
  	\end{itemize}
    

\end{frame}

%==========================================================

\section{Contraintes en performances}

\begin{frame}
  \frametitle{}

%On se base sur le cahier des charges de l'RNEA utilisé dans le filtrage dynamique du CoM
%--> [cube de contraintes]
%    - période d'échantillonnage 5ms (1 commande vectorielle $\mathbf{q}$ de 30 DOF) -> $1ms$ pour les calculs de dynamique
%    - fenêtre de filtrage dynamique 1.6s => 32 RNEA
%    => 1 RNEA en $30\mu s$
%    <v> ~ $4\mu s$ sur i7-4700HQ, ~ $15\mu s$ sur HRP2 Core2(TM) Duo E7500
%
%--> précision numérique 0.1%
%
%- Algèbre Spatiale et méta programmation
%
%  - basée sur les torseurs (avec différence de notation), réduit la compléxité de la modélisation dynamique, comb. translation/rotation
%    (composition simplifiée des vitesses et accélérations)
%  - Algos CRBA et RNEA exploitent la récursivité (\emph{pattern} Visiteur), la "sparsité" / abscesnce de couplage entre branches cinématiques
%  - méta prog. => propriétés d'un langage symbolique (calculs inutiles non exécutés)
%  - méta fonctions génèrent des traitements/calculs numériques à la compilation
%    (conditions méta fonction: params->types ou int statiques et constants)
%
%- Architecture de la librairie \emph{metapod}
%
%%<Schéma global figure 1.1 p12>
%
%\emph{metapo} fournit algos basés sur modèle dynamique général
%prend un modèle $M$ spécifique (header)
%génère du code optimisé spécifique au modèle $M$
%
%%<schéma fonctionnel interne de metapod>

\end{frame}

%==========================================================

\section{Formalisme: l'Algèbre Spatiale}

\begin{frame}
  \frametitle{}
  \framesubtitle{}
\end{frame}

%==========================================================

\part{}
\section{L'algorithme de Dynamique Hybride}

\subsection{Principes et implémentation}

\begin{frame}
  \frametitle{Etat initial de metapod et besoins}
 
  Algorithmes déjà implémentés:
  \begin{itemize}
  \item Dynamique inverse:	"Recursive Newton-Euler Algorithm"
  \item Dynamique directe:	"Composite-Rigid-Body Algorithm"
  \end{itemize}
  \bigskip
  Algorithmes à implémenter et intégrer à la Dynamique Hybride:
  \begin{itemize}
  \item Calcul optimisé de $H$
  \item Dynamique directe:	résolution de $H \boldsymbol{\ddot{q} = \tau - C}$
  \item Dynamique inverse différentielle: $\mathrm{ID}_{\delta}(\boldsymbol{\ddot{q}}) = \mathrm{ID}(\boldsymbol{\ddot{q}}) - \mathrm{ID}(0)$
  \end{itemize}
  
\end{frame}

\begin{frame}

  \frametitle{Equation de mouvement}
  
  \begin{block}{équation de mouvement d'un arbre cinématique:}
  \begin{equation} \label{equ_equationMvt}
	H\bsy{(q) \ddot{q} + C(q,\dot{q},f^{ext}) = \tau}
	\end{equation}
  \end{block}
  
  \begin{description}
    \item[$\bsy{q, \dot{q}, \ddot{q}}$ :] vecteurs de position, vitesse, accélération
    \item[$\bsy{\tau}$ :] forces/couples moteurs (internes)
    \item[$\bsy{H}$ :] matrice des termes inertiels
    \item[$\bsy{C}$ :] forces de précontrainte extérieures
  \end{description}
  
  \note{pour chaque articulation $i$ on connaît soit le couple soit l'accélération}
  \begin{equation*}
  q_i
  \begin{cases}
  \text{articulation $fd$ "forward dynamics"}: &\tau_i \text{ connu} \\
  \text{articulation $id$ "inverse dynamics"}: &\ddot{q}_i \text{ connu}
  \end{cases}
  \end{equation*}
  
  ordre par défaut de $q_i$ dans $\boldsymbol{q}$:
  $\hookrightarrow$ suivant parcours DFS
  
  \note{ordre pour tous les algorithmes}

\end{frame}

\begin{frame}

  \frametitle{Permutation des vecteurs et coefficients...}
  \setmyFiguresFile{hybridDynamics4etapes}
  \begin{columns}[T]
    \begin{column}{.7\textwidth}\small
	  \begin{align*}
		\ddot{q} &= 
		\begin{bmatrix}
		  \ddot{q}_1 & \ddot{q}_2 & \textcolor{red}{\ddot{q}_4} & \ddot{q}_5 & \textcolor{red}{\ddot{q}_3} & \ddot{q}_6 & \ddot{q}_7
		\end{bmatrix}^T \\
		Q \ddot{q} &= 
		\begin{bmatrix}
		  \textcolor{red}{\ddot{q}_4} & \textcolor{red}{\ddot{q}_3} & \ddot{q}_1 & \ddot{q}_2 & \ddot{q}_5 & \ddot{q}_6 & \ddot{q}_7
		\end{bmatrix}
		=
		\begin{bmatrix}
		  \underline{\ddot{q}_1} \\
		  \underline{\ddot{q}_2}
		\end{bmatrix} \\
		\textnormal{ et } &\textnormal{de même } \\
		Q \tau &= 
		\begin{bmatrix}
		  \underline{\tau_1} \\
		  \underline{\tau_2}
		\end{bmatrix}
		\: , \:
		Q C = 
		\begin{bmatrix}
		  C_1 \\
		  C_2
		\end{bmatrix}
		\: , \: Q H Q^T = 
		\begin{bmatrix}
		  H_{11} & H_{12} \\
		  H_{21} & H_{22}
		\end{bmatrix} 
		\end{align*}
  \end{column}%
  \hfill%
  \begin{column}{.3\textwidth}
    \dispFig[H]{1}{\textwidth}{Arbre cinématique}{fig_chdaArbreK1}
  \end{column}%
  \end{columns} \pause
	
	\begin{columns}[T]
	\begin{column}{.5\textwidth}\footnotesize
	\begin{alertblock}{équation de mouvement reformulée:}
	  \begin{align}
	  	\begin{bmatrix}
		  H_{11} & H_{12} \\
		  H_{21} & H_{22}
		\end{bmatrix} 
		\cdot
		\begin{bmatrix}
		  \ddot{q}_{1} \\
		  \ddot{q}_{2}
		\end{bmatrix} 
		= 
		\begin{bmatrix}
		  \tau_{1} \\
		  \tau_{2}
		\end{bmatrix} 
		-
		\begin{bmatrix}
		  C_{1} \\
		  C_{2}
		\end{bmatrix} \label{equ_local_eqMvt_2}
		\end{align}
	\end{alertblock}
	\end{column}

	\begin{column}{.5\textwidth}\footnotesize
	\begin{alertblock}{Inconnues rassemblées à gauche:}
		\begin{align}
		\begin{bmatrix}
		  H_{11} & 0 \\
		  H_{21} &  -I
		\end{bmatrix} 
		\cdot
		\begin{bmatrix}
		  \ddot{q}_1 \\
		  \tau_2
		\end{bmatrix} 
		=
		\begin{bmatrix}
		  \tau_1 \\
		  0
		\end{bmatrix} 
		-
		\begin{bmatrix}
		  C'_{1} \\
		  C'_{2}
		\end{bmatrix} \label{equ_equationMvt_dynHyb2} \\
		\notag \\
		{\tiny \textnormal{Avec} \qquad
		\begin{bmatrix}
		  C'_{1} \\
		  C'_{2}
		\end{bmatrix}
		=
		\begin{bmatrix}
		  C_{1} \\
		  C_{2}
		\end{bmatrix}
		+
		\begin{bmatrix}
		  H_{12} \ddot{q}_2 \\
		  H_{22} \ddot{q}_2
		\end{bmatrix}} \label{equ_cPrime}
		\end{align}
	\end{alertblock}
	\end{column}
	\end{columns}
  
\end{frame}

\begin{frame}\small

  \frametitle{Quatre étapes de résolution...}
  
  % affichage au coin droit supérieur des équations de référence
  \begin{columns}[T]
  \begin{column}{.6\textwidth}
    equ.\eqref{equ_equationMvt_dynHyb2} Se décline en 2 lignes... \bigskip \\
    \begin{enumerate}
    \item <3-> calcul de $\boldsymbol{C'}$
    \item <3-> calcul de $H_{11}$
    \item <1-> ${\eqref{equ_equationMvt_dynHyb2} \implies H_{11} \bsy{\ddot{q}_1 = \tau_1 - C'_1}}$ \\
    $\hookrightarrow$ résoudre $\ddot{q}_1$
    \item <2-> $\eqref{equ_equationMvt_dynHyb2} \implies \bsy{\tau_2 = C'_2} + H_{21} \bsy{\ddot{q}_1}$
    \end{enumerate}
  \end{column}
  \begin{column}{.37\textwidth}
  	\begin{block}{\footnotesize{équations de mouvement reformulées:}}\tiny
	  \begin{align*}
	  	\begin{bmatrix}
		  H_{11} & H_{12} \\
		  H_{21} & H_{22}
		\end{bmatrix} 
		\cdot
		\begin{bmatrix}
		  \ddot{q}_{1} \\
		  \ddot{q}_{2}
		\end{bmatrix} 
		&= 
		\begin{bmatrix}
		  \tau_{1} \\
		  \tau_{2}
		\end{bmatrix} 
		-
		\begin{bmatrix}
		  C_{1} \\
		  C_{2}
		\end{bmatrix} \quad \eqref{equ_local_eqMvt_2} \\
		\notag \\
		\begin{bmatrix}
		  H_{11} & 0 \\
		  H_{21} &  -I
		\end{bmatrix} 
		\cdot
		\begin{bmatrix}
		  \ddot{q}_1 \\
		  \tau_2
		\end{bmatrix} 
		&=
		\begin{bmatrix}
		  \tau_1 \\
		  0
		\end{bmatrix} 
		-
		\begin{bmatrix}
		  C'_{1} \\
		  C'_{2}
		\end{bmatrix} \quad \eqref{equ_equationMvt_dynHyb2} \\
		\end{align*}
	\end{block}
  \end{column}
  \end{columns}
  \vfill

\end{frame}

\begin{frame}
  \frametitle{Calcul des coefficients...}
  
  	\begin{block}{\footnotesize{équations de mouvement reformulées:}}\tiny
  	\setlength\abovedisplayskip{0pt}
  	\setlength\belowdisplayskip{0pt}
  \begin{align*}
  	\begin{bmatrix}
	  H_{11} & H_{12} \\
	  H_{21} & H_{22}
	\end{bmatrix} 
	\cdot
	\begin{bmatrix}
	  \ddot{q}_{1} \\
	  \ddot{q}_{2}
	\end{bmatrix} 
	= 
	\begin{bmatrix}
	  \tau_{1} \\
	  \tau_{2}
	\end{bmatrix} 
	-
	\begin{bmatrix}
	  C_{1} \\
	  C_{2}
	\end{bmatrix} \quad \eqref{equ_local_eqMvt_2} \hspace{2cm}
	\begin{bmatrix}
	  H_{11} & 0 \\
	  H_{21} &  -I
	\end{bmatrix} 
	\cdot
	\begin{bmatrix}
	  \ddot{q}_1 \\
	  \tau_2
	\end{bmatrix} 
	=
	\begin{bmatrix}
	  \tau_1 \\
	  0
	\end{bmatrix} 
	-
	\begin{bmatrix}
	  C'_{1} \\
	  C'_{2}
	\end{bmatrix} \quad \eqref{equ_equationMvt_dynHyb2} \\
	\end{align*}
  \end{block}
  
	Calcul de $C'$: 
	\begin{columns}[onlytextwidth]
	\begin{column}[c]{0.8\textwidth}\scriptsize
	\begin{align*}
	\left. \textnormal{Pour} \quad \ddot{q}=
	\begin{bmatrix}
	  0 \\
	  \ddot{q}_2
	\end{bmatrix}
	: 
	\begin{cases}
	\eqref{equ_equationMvt_dynHyb2}
	\implies
	\begin{bmatrix}
	  C'_1 \\
	  C'_2
	\end{bmatrix}
	=
	\begin{bmatrix}
	  \tau_1 \\
	  \tau_2
	\end{bmatrix} \\
	& \\
	\eqref{equ_local_eqMvt_2}
	\implies
	\begin{bmatrix}
	  \tau_1 \\
	  \tau_2
	\end{bmatrix}
	=
	Q \cdot \mathrm{ID} \left( q,\dot{q},Q^T
	\begin{bmatrix}
	  0 \\
	  \ddot{q}_2
	\end{bmatrix} \right)
	\end{cases} \right]
	\iff
	\end{align*}
	\end{column}
	\begin{column}[c]{0.27\textwidth}\tiny
	\begin{block}{}
	\begin{spacing}{1.5}
	$\begin{bmatrix}
	  C'_1 \\
	  C'_2
	\end{bmatrix}
	=
	Q \cdot \mathrm{ID} \left( q,\dot{q},Q^T
	\begin{bmatrix}
	  0 \\
	  \ddot{q}_2
	\end{bmatrix} \right)$
	\end{spacing}
	\end{block}
	\end{column}
	\end{columns}
  
  \bigskip
	Calcul de $H_{11}$, sous-matrice de $H$:
	\begin{itemize}
	\item calcul de $H = \mathrm{CRBA}(modèle,q)$
	\item permutation $H' = Q H Q^T$
	\item sélection de $H_{11}$ (méthode d'accès par blocs de la classe \verb;Eigen::Matrix; $\longrightarrow$ \verb;H'.block<$n_{fd}$,$n_{fd}$>;)
	\end{itemize}

\end{frame}

\begin{frame}[allowframebreaks]
  \frametitle{Résolution des équations...}
  
  \begin{columns}[T]
	\begin{column}{0.5\textwidth}
		Résoudre $\ddot{q}_1$:
		\begin{itemize}
		\item Système linéaire: $H_{11} \ddot{q}_1 = \tau_1 - C'_1$
		\item Inversion de $H$ trop coûteuse \\
		      (complexité $O(n^3)$ \\
		      $\hookrightarrow$  décompoer de $H$
		\end{itemize}
  \end{column}
	\begin{column}{0.5\textwidth}\scriptsize
		\begin{tabular}[H]{|l|c|}
		\hline
		Décomposition ou méthode & Complexité $O$ \\ \hline \hline
		inversion directe de matrice & $O(n^3)$ \\ \hline
		LLT et LDLT & $O(n^3/3)$ \\ \hline
		LU & $O(2n^3/3)$ \\ \hline
		QR & $O(4n^3/3)$ \\
		\hline
		\end{tabular}
	\end{column}
	\end{columns}

  \bigskip
  
  \begin{columns}[T]
	\begin{column}{0.55\textwidth}
		Critères de choix de la décomposition: propriétés de $H_{11}$ et complexité de la décomposition:
		\begin{itemize}
		\item $H$ et $H_{11}$ symétriques, définies positives 
		\item factorisation $LDL^T$ et $LL^T$ sont applicables
		\item $\hookrightarrow$ factorisation $LDL^T$ (robuste, rapide, la plus appropriée)
		\end{itemize}
  \end{column}
	\begin{column}{0.45\textwidth}
	  \begin{alertblock}{décomposition de $H_{11}$}
    $H_{11} = L D L^T$
    \begin{itemize}
	    \item $D \longrightarrow$ diagonale
	    \item $L \longrightarrow$ triangulaire inférieure \\
	                              $\forall i: L_{ii}=1$)
	    \item Solveur \verb;Eigen::LDLT;
	  \end{itemize}
    \end{alertblock}
	\end{column}
	\end{columns} \vfill

	\framebreak
	
	Calcul de $\tau_2$:
	\begin{itemize}
	\item Sélection de $H_{21}$ dans  $H'$
	\item $\tau_2 = C'_2 + H_{21} \ddot{q}_1$
	\end{itemize} \vfill
	
\end{frame}
	
\begin{frame}
	\frametitle{Algorithme complet:}
	  
	\begin{spacing}{1.5}
	\begin{columns}[T]\scriptsize
%	\setlength{\columnsep}{10pt}
%	\setlength{\columnseprule}{0.5pt}
	\begin{column}{.48\textwidth}
	\begin{pseudocode}{hybridDynamics}{model, q, \dot{q}, \ddot{q}, \tau}
	(1)
	\BEGIN
	  \textnormal{\# Calul de $C'$, $C'_1$ et $C'_2$} \\
	  \ddot{q}_{perm} \GETS Q \ddot{q} \\
	  (\ddot{q}_{perm})_{i \in [1,fd]} \GETS 0 \\
	  \ddot{q}_2 \GETS (\ddot{q}_{perm})_{i \in [fd+1,n_{dof}]} \\
	  \ddot{q} \GETS Q^T \ddot{q}_{perm} \\
	  model.torques \GETS rnea(model,q,\dot{q},\ddot{q}) \\
	  C' \GETS getTorques(model) \\
	  C'_{perm} \GETS Q C'\\
	  C'_1 \GETS (C'_{perm})_{i \in [1,fd]} \\
	  C'_2 \GETS (C'_{perm})_{i \in [fd+1,n_{dof}]}
	\END \\
	(2)
	\BEGIN
	  \textnormal{\# Calcul de $H_{11}$} \\
	  model.H \GETS crba(model,q) \\
	  H_{perm} \GETS Q model.H Q^T \\
	  H_{11} \GETS (H_{perm})_{i,j \in [1,fd]} 
	\END 
	\end{pseudocode}
	\end{column}
	\begin{column}{.48\textwidth}
	\begin{pseudocode}{ }{ }
	(3) 
	\BEGIN
	  \textnormal{\# Résolution de l'équation $H_{11} \ddot{q}_1 = \tau_1 - C'_1$} \\
	  \tau_{perm} \GETS Q \tau \\
	  \tau_1 \GETS (\tau_{perm})_{i \in [1,fd]} \\
	  \tau_2 \GETS (\tau_{perm})_{i \in [fd+1,n_{dof}]} \\
	  b \GETS \tau_1 - C'_1 \\
	  A \GETS H_{11} \\
	  lltOfH11 \GETS LLT(A) \hfill \textnormal{\# création du solveur} \\
	  \ddot{q}_1 \GETS lltOfH11.solve(b) \hfill \textnormal{\# résolution} 
	\END \\
	(4) 
	\BEGIN
	  \textnormal{\# Calcul de $\tau_2$ et reconstruction des vecteurs de} \\
	  \textnormal{\# sortie $\ddot{q}$ et $\tau$} \\
	   H_{21} \GETS (QHQ^T)_{n_{fd+1} \leqslant i \leqslant n_{dof},1 \leqslant j \leqslant n_{fd}} \\
	  \tau_2 \GETS C'_2 + H_{21} \ddot{q}_1 \\
	  \tau \GETS Q^T \begin{bmatrix} \tau_1 & \tau_2 \end{bmatrix}^T \\
	  \ddot{q} \GETS Q^T \begin{bmatrix} \ddot{q}_1 & \ddot{q}_2 \end{bmatrix}^T
	\END 
	\end{pseudocode}
	\end{column}
	\end{columns}
	\end{spacing}
  \vfill
	  
\end{frame}

%==========================================================


\subsection{Optimisations}

\begin{frame}
  \frametitle{Optimisation du CRBA et des transformations $^sX_p$}
  
  \begin{block}{Correction du calcul optimal de $^sXp$}
  \begin{itemize}
	  \item rotations $X_J$: axe parallèle à $Ox$ ou $Oy$ ou $Oz$ du repère lié au corps par $X^T$
	  \item $\hookrightarrow$ calcul des transformations $^sXp$ dans \verb;jcalc;
	  \item $\hookrightarrow$ calcul des vitesses et accélérations composées dans \verb;jcalc;.
  \end{itemize}
  \end{block}
  
  \begin{block}{Calcul de $H$ limité aux coefficients de $H_{11}$ (Hybrid CRBA)}
  \begin{itemize}
    \item Calcul de $H$ complet: $\forall noeud_i$ sauf la base, calcul de $H_{ij}$ ($j$ = noeud sélectionné en remontant l'arbre vers la racine)
	  \item Optimisation $H_{11}$: parcours de $\nu(fd)$, et $H_{ij}$ calculés pour $i,j \in fd$
	  \item Optimisation $H_{11}$ + $H_{21}$: parcours étendu au reste de l'arbre si $i \in fd$
  \end{itemize}
	\end{block}    
  
\end{frame}

\setmyFiguresFile{optimisations}

\begin{frame}
  \frametitle{Définitions...}
  
  	\begin{columns}
	\begin{column}{.48\textwidth}
  \dispFig[T!]{3}{\textwidth}{calcul de $H_{ij}$}{chdaCalculHij} \vfill
  \end{column}
  \begin{column}{.48\textwidth}
  \vfill \dispFig[B!]{2}{.8\textwidth}{Inertie composite ${I_i}^c$}{chdaCalculIic}
  \end{column}
  \end{columns}
  
\end{frame}

\begin{frame}
  \frametitle{Quelques exemples pour le CRBA optimisé (Hybrid CRBA)}
  
	\begin{figure}[H]
	  \begin{center}
%   \begin{overprint}
	  \includegraphics[width=\textwidth, page=4]{figs/\myFiguresFile}
	  \caption{Calcul de $H$ complet}
	  \includegraphics[width=.9\textwidth, page=6]{figs/\myFiguresFile}
	  \caption{Optimisation $H_{11}$}
%   	\end{overprint}
	  \end{center}
	\end{figure}
	
\end{frame}

\begin{frame}
	\begin{figure}[H]
	  \begin{center}
	  \includegraphics[width=.9\textwidth, page=6]{figs/\myFiguresFile}
	  \caption{Optimisation $H_{11}$}
	  \includegraphics[width=.9\textwidth, page=7]{figs/\myFiguresFile}
	  \caption{Optimisation $H_{11}$ + $H_{21}$}
	  \end{center}
	\end{figure}
	
\end{frame}

\begin{frame}
  \frametitle{Réduction des dépendances et calcul optimal de $\tau_2$}

  Dynamique Inverse Différentielle pour le calcul de $\tau$:
	\begin{equation}\footnotesize
	\tau = \bsy{C'} + \mathrm{ID}_{\delta} \left( Q^T \bsy{\begin{bmatrix} 
	                                                         \ddot{q}_1 \\
	                                                         0 
	                                                       \end{bmatrix}} \right) \label{equ_tauIDdiff_2}
	=
	\bsy{C'} + \mathrm{ID} \left( \bsy{q,\dot{q}},Q^T \bsy{\begin{bmatrix}
	                                                         \ddot{q}_1 \\      
	                                                          0         
	                                                       \end{bmatrix}} \right) - Q \: \mathrm{ID}(\bsy{q,\dot{q},0})
	\end{equation}
	
	\bigskip
	
	\begin{itemize}
	\item[$\hookrightarrow$] simplification des termes en $\bsy{q, \dot{q}}$ (RNEA simplifié)
	\item[$\hookrightarrow$] abandon de $H_{21}$ (HCRBA accéléré)
	\end{itemize}
  
\end{frame}

\begin{frame}\scriptsize
  \frametitle{Résultats: mesure de durée d'exécution}

  \note{Mesure des temps d'exécutions relatifs à un RNEA classique
        Mesures du temps d'exécution de chacune des étapes (a -> e) de l'algorithme (4 + reconstruction des vecteurs)
        Comparaison des optimisations (1 -> 4)}
  Optimisations:
	\begin{enumerate}
	\item matrices de passage à axes fixes prédéfinis
	\item HCRBA(CRBA hybride) $H_{11}-H_{12}-H_{21}$
	\item utilisation du module \emph{Eigen} de Sparsité
	\item HCRBA $H_{11}$ seule + Dynamique Inverse différenciel ($ID_\delta$), sans la sparsité \emph{Eigen}
	\end{enumerate}
	
	\note{les optimisations sont cumulées dans l'ordre de numérotation. Voici l'ensemble des mesures réalisées sur notre modèle de robot humanoïde}
	
	\begin{flushleft}
	
	\begin{table}[H]
	\begin{center}
	\begin{tabular}[H]{|l|l|}
	\hline
	Nom du CPU & Intel(R) Core(TM) i7-4700HQ CPU \\ \hline \hline
	Fréquence & 2.40GHz \\ \hline
	Fréquence bus CPU & 800 MHz \\ \hline
	Nombre de coeurs & 8(4) \\ \hline
	Taille du cache & 6144 Kb \\ \hline
	Distribution & Ubuntu 14.04 LTS (trusty) \\
	\hline
	\end{tabular}
	\caption[Table caption text]{Propriétés du processeur de la plateforme de test}
	\label{table:propriétésProc}
	\end{center}
	\end{table}
	
	\end{flushleft}
	
\end{frame}

\begin{frame}\scriptsize
  
	\begin{flushleft}
  
	\begin{table}[H]
	\begin{center}
	\begin{tabular}[H]{|l|l|l|}
	\hline
	Version d'implémentation                   & implémentation initiale    & (1)      \\ \hline \hline
	Durée moyenne                              & 16.286723                  & 14.07054 \\
	a: RNEA                                    & 6.39608                    & 5.46801  \\
	b: CRBA                                    & 7.59427                    & 6.31883  \\
	c: $\ddot{q}_1$ (solver)                   & 0.894731                   & 0.872069 \\
	d: $\tau_2$                                & 0.730014                   & 0.734224 \\
	e: reconstruction de $\tau$ et $\ddot{q}$  & 0.671628                   & 0.677407 \\
	\hline
	\end{tabular}
	\caption[Table caption text]{Durées de traitement (en $\mu s$) des étapes du CHDA: optimisation liée aux repères à axes fixes.}
	\label{table:performancesOptimAxesFixes}
	\end{center}
	\end{table}
	
	\begin{table}[H]
	\begin{center}
	\begin{tabular}[H]{|l|l|l|l|l|l|l|l|l|}
	\hline
	Version d'implémentation                   & \multicolumn{2}{c|}{(1)}  & \multicolumn{2}{c|}{(2)} & \multicolumn{2}{c|}{(3)} & \multicolumn{2}{c|}{(4)} \\ \hline \hline
	                                           & $\mu s$      & (\%)      & $\mu s$     & (\%)   & $\mu s$     & (\%)   & $\mu s$   & (\%) \\ \hline
	Durée moyenne                              & 14.07054     & 2.57      & 11.118823   & 2.16   & 11.626056   & 2.16   & 10.968547 & 2.02 \\
	a: RNEA                                    & 5.46801      & 1         & 5.14257     & 1      & 5.38397     & 1      & 5.42778   & 1    \\
	b: CRBA                                    & 6.31883      & 1.16      & 3.66517     & 0.71   & 3.61689     & 0.67   & 2.3934    & 0.44 \\
	c: $\ddot{q}_1$ (solver)                   & 0.872069     & 0.16      & 0.889762    & 0.17   & 1.21743     & 0.23   & 0.886647  & 0.16 \\
	d: $\tau_2$                                & 0.734224     & 0.13      & 0.738075    & 0.14   & 0.731042    & 0.14   & 2.26072   & 0.42 \\
	e: reconstruction de $\tau$ et $\ddot{q}$  & 0.677407     & 0.12      & 0.683246    & 0.13   & 0.676724    & 0.13   & 0         & 0    \\
	\hline
	\end{tabular}
	\caption[Table caption text]{Durées de traitement (en $\mu s$) des étapes du CHDA: les optimisationsspécifiques au CHDA. On a reporté, pour chacune d'elles le gain relatif par rapport à un traitement de référence qui est l'étape 1 (RNEA classique).}
	\label{table:performancesOptimSpecifCHDA}
	\end{center}
	\end{table}
	
	\end{flushleft}
	
	
\end{frame}

%==========================================================
\section{Méthodes}

\subsection{Méthodes de développement et de validation}

\begin{frame}
  \frametitle{}
  
  \setmyFiguresFile{cycleDev}
  \dispFig[H]{1}{.8\textwidth}{Cycle de développement}{}
  
\end{frame}

\begin{frame}

  tests unitaires et Validation:
  \begin{itemize}
		\item exécution pas à pas des fonctions créées (gdb, qtcreator)
		\item validation de résultats d'algorithmes: traitements alternatifs (HCRBA, RNEA/CRBA)
		\item Méta fonctions: comparaison phase compilation vs exécution
		\item Matrice de permutation: vérification fonctionnelle et des propriétés 
		\item Génération aléatoire de vecteurs de test $q$, $\dot{q}$, $\ddot{q}$, $\tau$
		\item Mesures de temps d'exécution sur 100000 itérations
  \end{itemize}

\end{frame}

%==========================================================

\subsection{Outils de développement}

\begin{frame}
  \frametitle{}
  Langages et outils de développement:
  \begin{itemize}
  \item template C++ et Méta programmation
  \item BOOST : méthode de test $\curvearrowright$ suite de tests unitaires \emph{metapo} \\
                création aisée de tableaux de types ou d'objets templatés
  \item Eigen
  \item cmake
  \end{itemize}
  
  \bigskip
  Outils d'édition, analyse et gestion de configuration:\\
  emacs, Qtcreator, gdb, github, git, gitk, cachegrind
    
\end{frame}

%==========================================================

\section{Conclusion et perspectives}

\begin{frame}
  \frametitle{}
  
  \huge{CONCLUSION}
  
\end{frame}


\end{document}
