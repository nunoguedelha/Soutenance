\documentclass[11pt]{beamer}
\usetheme{Darmstadt}
%\usecolortheme{beaver}

%Include all packages
\ProvidesPackage{./tex/mySlideShowStyle}



%%%%%%%%%%%%%%%%%
\author{Nuno Guedelha}
\title{Implémentation et application d'un algorithme de Dynamique Hybride}
%\setbeamercovered{transparent} 
%\setbeamertemplate{navigation symbols}{} 
%\logo{} 
\institute{LAAS-CNRS}
\date{} 
%\subject{} 

\setbeamertemplate{footline}{
\leavevmode%
\hbox{\hspace*{-0.06cm}
\begin{beamercolorbox}[wd=.2\paperwidth,ht=2.25ex,dp=1ex,center]{author in head/foot}%
	\usebeamerfont{author in head/foot}\insertshortauthor%~~(\insertshortinstitute)
\end{beamercolorbox}%
\begin{beamercolorbox}[wd=.55\paperwidth,ht=2.25ex,dp=1ex,center]{section in head/foot}%
	\usebeamerfont{section in head/foot}\insertshorttitle
\end{beamercolorbox}%
\begin{beamercolorbox}[wd=.25\paperwidth,ht=2.25ex,dp=1ex,right]{section in head/foot}%
	\usebeamerfont{section in head/foot}\insertshortdate{}\hspace*{2em}
	\insertframenumber{} / \inserttotalframenumber\hspace*{2ex}
\end{beamercolorbox}}%
\vskip0pt%
}

\begin{document}

\begin{frame}
\titlepage
\end{frame}
%==========================================================

\begin{frame}
\tableofcontents
\end{frame}
%==========================================================

\section{Introduction}

\begin{frame}
  \frametitle{Contexte du stage}
  \framesubtitle{}
  \begin{block}{Projet professionnel}
  \begin{itemize}
    \item première expérience dans l'embarqué et le temps réel
    \item nouveau cap: la robotique mobile et le Master 2 IRR
    \item Objectif: chercheur en Robotique (entreprise ou laboratoire)
  \end{itemize}
  \end{block}
  \begin{block}{Le LAAS et Gepetto}
  \begin{itemize}
    \item Gepetto: expertise en analyse, génération de mouvements  
    \item Approche fondamentale, modélisation dynamique, contrôle, planification de mouvements
    \item {Intégration dans des packages open source \\
          $\Rsh$ metapod: librairie de modélisation dynamique}
    \note{}
  \end{itemize}
  \end{block}
\end{frame}

\begin{frame}
  \frametitle{Objectifs}
  \framesubtitle{}
\end{frame}
%==========================================================

\section{Modélisation Dynamique.}

%==========================================================

\section{Contraintes en performances}

%==========================================================

\section{Formalisme de l'Algèbre Spatiale}

%==========================================================

\section{Algorithme de Dynamique Hybride}

\subsection{principes et implémentation}

\begin{frame}
  \frametitle{Etat initial de metapod et besoins}
 
  Algorithmes déjà implémentés:
  Dynamique inverse:	Recursive Newton-Euler Algorithm
  Dynamique directe:	Composite-Rigid-Body Algorithm
  
  Algorithmes à implémenter et intégrer à la Dynamique Hybride:
  Calcul optimisé de $H$
  Dynamique directe:	résolution de $H \ddot{q}=\tau - C$
  Dynamique inverse différentielle
  
\end{frame}

%\begin{frame}
%
%  \frametitle[allowframebreaks]{Equation de mouvement}
%
%  équation de mouvement d'un arbre cinématique:
%  
%  <equ 3.1>
%  
%  <description des variables>
%  
%  (pour chaque articulation $i$ on connaît soit le couple soit l'accélération)
%  articulations\emph{fd} "forward dynamics" => $\tau_i$ connu
%  articulation \emph{id} "inverse dynamics" => $\ddot{q}_i$ connu
%  
%  ordre par défaut (pour tous les algorithmes) de $q_i$ dans $\mathbf{q}$ (exemple):
%  
%  <figure 3.1>
%  
%  => ordre du parcours en profondeur DFS:
%  
%\begin{align*}
%\ddot{q} &= 
%\begin{bmatrix}
%  \ddot{q}_1 & \ddot{q}_2 & \textcolor{blue}{\ddot{q}_4} & \ddot{q}_5 & \textcolor{blue}{\ddot{q}_3} & \ddot{q}_6 & \ddot{q}_7
%\end{bmatrix}^T \\
%Q \ddot{q} &= 
%\begin{bmatrix}
%  \textcolor{blue}{\ddot{q}_4} & \textcolor{blue}{\ddot{q}_3} & \ddot{q}_1 & \ddot{q}_2 & \ddot{q}_5 & \ddot{q}_6 & \ddot{q}_7
%\end{bmatrix}
%=
%\begin{bmatrix}
%  \underline{\ddot{q}_1} \\
%  \underline{\ddot{q}_2}
%\end{bmatrix} \\
%\textnormal{ et de même } \\
%Q \tau &= 
%\begin{bmatrix}
%  \underline{\tau_1} \\
%  \underline{\tau_2}
%\end{bmatrix}
%\quad \textnormal{et} \quad
%Q C = 
%\begin{bmatrix}
%  C_1 \\
%  C_2
%\end{bmatrix}
%\quad \textnormal{et} \quad
%Q H Q^T = 
%\begin{bmatrix}
%  H_{11} & H_{12} \\
%  H_{21} & H_{22}
%\end{bmatrix} 
%\end{align*}
%
%  Reformulation de l'équation de mouvement (3.2 => inconnues rassemblées à gauche):
%  
%  <equ 3.5> \label{equ_equationMvt_dynHyb1}
%  <equ 3.2 et 3.3> \label{equ_equationMvt_dynHyb2}
%  
%\end{frame}
%
%\begin{frame}
%
%  \frametitle{Quatre étapes de résolution...}
%  
%  \eqref{equ_equationMvt_dynHyb2} Se décline en 2 lignes...
%  
%  $H_{11} \ddot{q}_1 = \tau_1 - C'_1$ --> résoudre $\ddot{q}_1$ \\
%\begin{align*}
%&H_{21} \ddot{q}_1 - \tau_2 = -C'_2
%\iff
%&\tau_2 = C'_2 + H_{21} \ddot{q}_1
%\end{align*}
%
%(introduire les deux premières étapes et numéroter 1-4)
%
%\end{frame}
%
%\begin{frame}
%  \frametitle{Calcul des coefficients...}
%  
%(garder au coin haut supérieur les 2 équations de mouvement)
%
%Calcul de $C'$: 
%\begin{align*}
%\textnormal{Pour} \quad \ddot{q}=
%\begin{bmatrix}
%  0 \\
%  \ddot{q}_2
%\end{bmatrix}
%: 
%&\left{
%\eqref{equ_equationMvt_dynHyb2}
%\iff
%\begin{bmatrix}
%  C'_1 \\
%  C'_2
%\end{bmatrix}
%=
%\begin{bmatrix}
%  \tau_1 \\
%  \tau_2
%\end{bmatrix} \\
%\eqref{equ_equationMvt_dynHyb1}
%\iff
%\begin{bmatrix}
%  C'_1 \\
%  C'_2
%\end{bmatrix}
%=
%Q \mathrm{ID} \left( q,\dot{q},Q^T
%\begin{bmatrix}
%  0 \\
%  \ddot{q}_2
%\end{bmatrix} \right)
%\right.
%\iff
%&\begin{bmatrix}
%  C'_1 \\
%  C'_2
%\end{bmatrix}
%=
%Q \mathrm{ID} \left( q,\dot{q},Q^T
%\begin{bmatrix}
%  0 \\
%  \ddot{q}_2
%\end{bmatrix} \right)
%\end{align*}
%
%Calcul de $H_{11}$ sous-matrice de $H$
%  - calcul de $H = \mathrm{CRBA}(modèle,q)$
%  - permutation $H' = Q H Q^T$
%  - sélection de $H_{11}$ (méthode d'accès par blocs de la classe \verb;Eigen::Matrix; => \verb;H'.block<n_{fd},n_{fd};)
%
%\end{frame}
%
%\begin{frame}
%  \frametitle{Résolution des équations...}
%  
%Résoudre $\ddot{q}_1$ \\
%  Système linéaire: $H_{11} \ddot{q}_1 = \tau_1 - C'_1$
%  Inversion de $H$ trop coûteuse (complexité $O(n^3)$ (où n est la dimension de $H$)
%  => décomposition de $H$
%  
%Choix de la décomposition:
%  $H$ et $H_{11}$ symétriques, définies positives 
%  [tableau avec complexité des diff. méthodes]
%\setlength{\intextsep}{0pt}
%\setlength{\columnsep}{0cm}
%\begin{wraptable}[7]{r}{0.45\textwidth}\raggedleft
%\begin{tabular}[H]{|l|c|}
%\hline
%Décomposition ou méthode & Complexité $O$ \\ \hline \hline
%inversion directe de matrice & $O(n^3)$ \\ \hline
%LLT et LDLT & $O(n^3/3)$ \\ \hline
%LU & $O(2n^3/3)$ \\ \hline
%QR & $O(4n^3/3)$ \\
%\hline
%\end{tabular}
%\end{wraptable}
%  => factorisation LDL^T (robuste, rapide, la plus appropriée):
%  $H_{11} = L D L^T$
%  ($D$ matrice diagonale,
%   $L$ matrice triangulaire inférieure et $\forall i: L_{ii}=1$)
%  Solveur \verb;Eigen::LDLT;
%
%Calcul de $\tau_2$:
%  Sélection de $H_{21}$ dans  $H'$
%  $\tau_2 = C'_2 + H_{21} \ddot{q}_1$
%
%\end{frame}
%
%\begin{frame}
%  \frametitle{Algorithme complet:}
%  
%  <algo 3.1.1>
%  
%\end{frame}
%==========================================================

\begin{frame}
  \frametitle{}
  \framesubtitle{}
\end{frame}
%==========================================================

\begin{frame}
  \frametitle{}
  \framesubtitle{}
\end{frame}
%==========================================================

\begin{frame}
  \frametitle{}
  \framesubtitle{}
\end{frame}
%==========================================================

%==========================================================

\subsection{Optimisations}

%==========================================================
\section{Méthodes et outils}

\subsection{Méthodes de développement et de validation}

%==========================================================

\subsection{Outils de développement}

%==========================================================

\section{Conclusion}



\end{document}